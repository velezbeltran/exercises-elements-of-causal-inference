\documentclass[11pt]{article}
\usepackage{geometry}
\usepackage{amsmath}
\usepackage{amssymb}
\usepackage{amsthm}
\usepackage{graphicx}
\usepackage{float}
\usepackage{hyperref}
\usepackage[english]{babel} % change 'english' to your language
\usepackage[utf8]{inputenc}
\usepackage[T1]{fontenc}
\usepackage{natbib} % or \usepackage{biblatex}
\usepackage{listings}
\usepackage{caption}
\usepackage{booktabs}
\usepackage{tabularx}
\usepackage{tikz}
\usepackage{mathtools}
\usepackage{fancyhdr}
\usepackage{enumerate}




\title{Chapter 3 Solution}
\begin{document}
\maketitle

\section*{Problem 3.6}
Take the model 
\begin{align*}
    C &= N_C \\
    E &= \alpha C + N_E
\end{align*}
What is the distribution of $P(C | E = 2)$
\subsection*{Solution}
Using bayes rule we have that 
\begin{align*}
P(C | E = 2) &= \frac{P(E = 2 | C) P(C)}{P(E = 2)} \\
&\propto P(E = 2 | C) P(C) \\
&\propto \exp\left( -\frac{1}{2} (2 - \alpha C)^2 \right) \exp\left(-\frac{1}{2} C^2\right) \\
&\propto \exp\left( -\frac{1}{2} (4 - 4 \alpha C +  (\alpha^2 + 1) C^2) \right) \\
&\propto \exp\left( -\frac{1}{2} (- 4 \alpha C +  (\alpha^2 + 1) C^2) \right) \\
&\propto \exp\left( -\frac{\alpha ^2 + 1}{2 } (- 4 \frac{\alpha}{\alpha^2 + 1} C +  C^2) \right) \\
&\propto \exp\left( -\frac{\alpha ^2 + 1}{2 } \left(C -  \frac{2\alpha }{\alpha^2 + 1}\right)^2\right) \\
\end{align*}
Clearly, minus the normalizing constant, this is the pdf of a normal distribution with
 mean $\mu = \frac{2\alpha }{\alpha^2 + 1}$ and variance $\sigma^2 = \frac{1}{\alpha^2 + 1}$.
 Therefore, the distribution must be a normal with these parameters.
 And plugging back in $\alpha = 4$ we get the result in the book. Namely 
\[P(C | E = 2) = \mathcal{N}(\mu = 8/17, \sigma^2 = 1/17)\]

\section*{Problem 3.7}
See the book
\subsection*{Solution}
We can intervene on X with the distribution $P(X) = \text{Bernoulli}(0.5)$ and then we can check the distribution of $Y$.
If it is normal then we know that the model $Y \rightarrow X$ is correct. 
If it is not normal then we know that the model $X \rightarrow Y$ is correct.

\section*{Problem 3.8}
See the book
\subsection*{Solution}
\begin{enumerate}[a)]
    \item 
    We need to find $\alpha, \beta, \gamma, \delta$ such that for every $n_x, n_y$ the equation holds. 
    Plugging in $X = \alpha n_x + \beta n_y$ and $Y = \gamma n_x + \delta n_y$ we get that our solution must satisfy
    \begin{align*}
        \alpha n_x + \beta n_y &=  2 \gamma n_x + 2 \delta n_y  + n_x \\
        \gamma n_x + \delta n_y &=  2 \alpha n_x + 2 \beta n_y  + n_y
    \end{align*}
    As this needs to hold for all $n_x$ then we get the following system of equations
    \begin{align*}
        \alpha &= 2 \gamma + 1 \\
        \beta &= 2 \delta \\
        \gamma &= 2 \alpha \\
        \delta &= 2 \beta + 1
    \end{align*}
    Which we can solve as follows. Plugging in 3 into  1 we get 
    \[ \alpha = 4 \alpha + 1 \Rightarrow \alpha = -1/3\]
    Plugging 2 in 4 we get
    \[ \delta = 4 \delta + 1 \Rightarrow \delta = -1/3\]
    and then using these two results in 2 and 3 we get
    \begin{align*}
        \beta &= 2 \delta = -2/3 \\
        \gamma &= 2 \alpha = -2/3
    \end{align*}

    So to summarize we have that
    \begin{align}
        \alpha &= -1/3 \\
        \beta &= -2/3 \\
        \gamma &= -2/3 \\
        \delta &= -1/3
    \end{align}
\end{enumerate}

% Repeat for more problems

\end{document}
